%----------------------------------------------------------------------------------------
%   USEFUL COMMANDS
%----------------------------------------------------------------------------------------

\newcommand{\dipartimento}{Dipartimento di Matematica ``Tullio Levi-Civita''}

%----------------------------------------------------------------------------------------
% 	USER DATA
%----------------------------------------------------------------------------------------

% Data di approvazione del piano da parte del tutor interno; nel formato GG Mese AAAA
% compilare inserendo al posto di GG 2 cifre per il giorno, e al posto di 
% AAAA 4 cifre per l'anno
\newcommand{\dataApprovazione}{Data}

% Dati dello Studente
\newcommand{\nomeStudente}{Davide}
\newcommand{\cognomeStudente}{Pivato}
\newcommand{\matricolaStudente}{1187595}
\newcommand{\emailStudente}{davide.pivato.1@studenti.unipd.it}
\newcommand{\telStudente}{+ 39 3467463805}

% Dati del Tutor Aziendale
\newcommand{\nomeTutorAziendale}{Roberta}
\newcommand{\cognomeTutorAziendale}{Mazza}
\newcommand{\emailTutorAziendale}{r.mazza@synclab.it}
\newcommand{\telTutorAziendale}{+ 39 328 659 4110}
\newcommand{\ruoloTutorAziendale}{}

% Dati dell'Azienda
\newcommand{\ragioneSocAzienda}{Sync Lab S.r.l.}
\newcommand{\indirizzoAzienda}{Galleria Spagna, 28, 35129 Padova PD}
\newcommand{\sitoAzienda}{https://www.synclab.it
}
\newcommand{\emailAzienda}{info@synclab.it}
\newcommand{\partitaIVAAzienda}{P.IVA 12345678999}

% Dati del Tutor Interno (Docente)
\newcommand{\titoloTutorInterno}{Prof.}
\newcommand{\nomeTutorInterno}{Luigi}
\newcommand{\cognomeTutorInterno}{De Giovanni}

\newcommand{\prospettoSettimanale}{
     % Personalizzare indicando in lista, i vari task settimana per settimana
     % sostituire a XX il totale ore della settimana
    \begin{itemize}
        \item \textbf{Prima Settimana (40 ore)}
        \begin{itemize}
            \item Incontro con le persone coinvolte nel progetto per discutere i requisiti e le richieste relativamente al progetto da sviluppare;
            \item Verifica strumenti di lavoro e condivisione del materiale, e strumenti di gestione degli avanzamenti;
            \item Studio strumenti di Data Visualization ed acquisizione delle funzioni di base.
        \end{itemize}
        \item \textbf{Seconda Settimana (20 ore)} 
        \begin{itemize}
            \item Approfondimento delle principali modalità di rappresentazione dei dati in ambito Dashboarding e Analytics (tipologie di grafici e loro utilizzo, tipi di indicatori più comuni quali KPI e KRI, marking, drill-down, geolocalizzazione ecc.);
            \item Studio ed approfondimento del caso di studio.
        \end{itemize}
        \item \textbf{Terza Settimana (40 ore)} 
        \begin{itemize}
        	\item Definizione e progettazione delle analisi e delle visualizzazioni di base da realizzare.
            \item Sviluppo ed implementazione analisi descrittiva di base in relazione al caso di studio.
        \end{itemize}
        \item \textbf{Quarta Settimana (40 ore)} 
        \begin{itemize}
            \item Studio dei moduli avanzati di Machine Learning ed Advanced Analytics disponibili sullo strumento di Data Visualization: supporto ad implementazioni di moduli embedded (in R o Python), geoanalytics, funzionalità integrate di data science e analisi predittiva;
            \item Definizione e progettazione delle analisi e delle visualizzazioni avanzate da realizzare;
            
        \end{itemize}
        \item \textbf{Quinta Settimana (40 ore)} 
        \begin{itemize}
            \item Implementazione Advanced Analytics in relazione al caso di studio.
        \end{itemize}
        \item \textbf{Sesta Settimana (40 ore)} 
        \begin{itemize}
            \item Completamento analisi avanzata;
            \item Analisi dei risultati;
            \item Eventuali interventi di approfondimento e riorganizzazione progettuale;
            \item Inizio stesura relazione.
        \end{itemize}
        \item \textbf{Settima Settimana (40 ore)} 
        \begin{itemize}
            \item Termine sviluppi e collaudo;
            \item Completamento relazione.
        \end{itemize}
        \item \textbf{Ottava Settimana - Conclusione (40 ore)} 
        \begin{itemize}
            \item Analisi finale dei risultati ottenuti;
            \item Valutazione dei vantaggi ottenuti rispetto all'analisi descrittiva di base;
            \item Valutazione delle problematiche introdotte dagli strumenti e dalle tecniche di Advanced Analytics;
            \item Perfezionamento e chiusura relazione.
            \item Conclusioni.
        \end{itemize}
    \end{itemize}
}

% Indicare il totale complessivo (deve essere compreso tra le 300 e le 320 ore)
\newcommand{\totaleOre}{}

\newcommand{\obiettiviObbligatori}{
	 \item \underline{\textit{O01}}: primo obiettivo;
	 \item \underline{\textit{O02}}: secondo obiettivo;
	 \item \underline{\textit{O03}}: terzo obiettivo;
	 
}

\newcommand{\obiettiviDesiderabili}{
	 \item \underline{\textit{D01}}: primo obiettivo;
	 \item \underline{\textit{D02}}: secondo obiettivo;
}

\newcommand{\obiettiviFacoltativi}{
	 \item \underline{\textit{F01}}: primo obiettivo;
	 \item \underline{\textit{F02}}: secondo obiettivo;
	 \item \underline{\textit{F03}}: terzo obiettivo;
}